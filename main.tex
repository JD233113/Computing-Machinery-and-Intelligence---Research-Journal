\documentclass{article}
\usepackage[utf8]{inputenc}

\title{Computing Machinery and Intelligence - Research Journal}
\author{JD233113}

\usepackage{natbib}
\usepackage{graphicx}

\begin{document}

\maketitle
\begin{flushleft}

\section{Introduction:}

Alan Turing's paper "Computing Machinery and Intelligence" (1950) begins with Turing considering the question "can machines think?". Turing quickly abandons this question due to the ambiguity of the terms used, as the definition of a machine and what it actually means to think are both highly debatable. This leads Turing to instead consider a new, more specific, question: could an electronic computer ever reliably succeed in a variation of The Imitation Game?
\break
\break
The Imitation Game involves 3 participants, referred to as "A", "B" and "C". A (a man) sits in a room alongside B (a woman) while C sits in another room. C must interrogate A and B through a series of typewritten questions and attempt to determine which of them is the woman by reading their responses (which are also typewritten). A aims to trick C into making an incorrect decision whereas B aims to help C to correctly identify her.\cite{turing1950_intelligence}. Turing's proposed variation of The Imitation Game has a digital computer take the place of A. In this variation, A's goal is to convince C that it is the real human. The use of typewritten questions/answers allows A and B to have no identifying features beyond their actual answers, avoiding the need for machines to emulate human appearance/voices in order to succeed at the test.
\break
\break
"Computing Machinery and Intelligence" also includes some of Turing's thoughts regarding learning machines - machines that simulate a child's brain but are then educated over time until they can reliably simulate an adult's brain. Turing states his belief that a program simulating a child's brain would likely be less difficult to create than a program simulating an adult brain as a child's brain is comparable to a notebook, having little mechanism involved but many blank sheets that can be filled in later.

\section{Context:}

Turing wrote "Computing Machinery and Intelligence" not only to put forward his own views but also to refute various existing arguments which claimed that machines cannot (or at least should not) think. These existing arguments had been made by several influential figures in computer science such as Geoffrey Jefferson and Ada Lovelace, so Turing believed it was important to provide refutations before his own arguments suggesting that machines can think would be taken seriously.
\break
\break
Turing responds to the following arguments, some against the general ability of machines to think and others which he saw as potential future arguments against the validity of his variation of The Imitation Game:
\begin{itemize}
\item The religious objection, which argues that thinking is a product of man's immortal soul, and that machines therefore cannot think as they lack souls.
\item The heads in the sand objection, which argues that the consequences of thinking machines would be dire and that they should therefore not be created.
\item Mathematical objections, which state that there are limits to what a computer based on logic can answer.
\item The argument from consciousness put forward by Geoffrey Jefferson in 1949. This argument effectively states that machines cannot think unless they can be driven to action by their own emotions and thoughts. For example, Jefferson states that a machine is not truly thinking until it can "...write a sonnet or compose a concerto because of thoughts and emotions felt, and not by the chance fall of symbols...".\cite{jefferson1949_BMJ}
\item Arguments from various disabilities, such as the inability of machines to fall in love or learn from experience.
\item Lady Lovelace's objection, which argues that machines are incapable of originality. Lovelace states that machines can have "...no pretensions whatever to originate anything" and can only "...do whatever we know how to order it to perform. It can follow analysis; but it has no power of anticipating any analytical relations or truths."\cite{lovelace1843_notes}
\item Arguments from continuity in the nervous system, which state that the physical differences between digital machines and the human brain are too great and that machines will therefore never be able to correctly simulate the human brain.
\item The argument from informality of behaviour, which states that as machines act based on rules they will always be predictable and thus not truly intelligent.
\item Arguments based on extra-sensory perception, which state that as machines are incapable of mind-reading/telepathy it will always be possible to tell them apart from humans in The Imitation Game.
\end{itemize}

\section{Contributions to Computer Science:}

Turing's variation of The Imitation Game has become widely known as the "Turing Test" and there has been much effort from computer scientists to create a program that can reliably succeed in it. As early as 1963 Joseph Weizenbaum created a program known as "ELIZA" that fooled several judges in a Turing Test but there is much debate as to whether ELIZA truly passed the test due to the judges involved having minimal knowledge of computing, or the capabilities of machines. Subsequent Turing Tests have therefore allowed only those with relevant backgrounds (such as computer scientists and philosophers) to act as judges. No program has reliably managed to pass the Turing Test to this day.
\break
\break
The Turing Test has inspired the creation of the Loebner Prize, an annual competition which offers prizes to the creators of computer programs that give answers most closely resembling human responses. The Loebner Prize judges have still not given out the largest reward which is intended for a program that fully passes the Turing Test.\cite{loebner_prize} Turing's initial prediction that the Turing Test would be passed by the year 2000 was incorrect but predictions continue to be made of when, if ever, a machine will be capable of passing it. For example, entrepreneur Mitchell Kapor has made a substantial bet against futurist Ray Kurzweil’s claim that a computer will be capable of passing the Turing Test by 2029.
\break
\break
The Turing Test has also inspired various other tests of both machine and human intelligence. A CAPTCHA (Completely Automated Public Turing test to tell Computers and Humans Apart) is effectively a "reverse Turing Test" in which a computer asks website visitors to perform various tasks to verify that they are human, such as correctly typing strings of distorted letters from an image.\cite{hasan2016_captcha} Another test inspired by the Turing Test is the Minimum Intelligent Signal Test which uses the same format as the original Turing Test but only allows the use of yes/no or true/false questions (with the aim of genuinely testing a machine's intelligence as opposed to its ability to emulate human speech patterns).\cite{mist2019}
\break
\break
Although "Computing Machinery and Intelligence" has been highly influential since it was first published, the adequacy of the Turing Test for representing a machine’s ability to think has been disputed by several papers. For example the philosopher John Searle argues that programs can pass the Turing Test without truly understanding any of the the language used, and that a machine capable of passing the Turing Test would always merely be imitating thought as opposed to genuinely thinking.\cite{searle1980_mbp}

\bibliographystyle{plain}
\bibliography{references}
\end{flushleft}
\end{document}
